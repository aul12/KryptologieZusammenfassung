\chapter{Zero Knowledge}
Ziel: Der Prover $P$ soll dem Verifier $V$ zeigen, dass er $x
$ (mit öffentlichem Gegenstück $\tilde{x}$) kennt ohne $x$ oder Informationen darüber 
zu Veröffentlichen.

Typischerweise funktioniert das Verfahren in drei Phase: Commitment, Challenge und 
Response. Die Wahrscheinlichkeit das ein betrügerischer Prover erkannt wird beträgt
$1/2$ die Fehlerwahrscheinlichkeit lässt sich durch Wiederholung beliebig drücken.

\section{Definition}
Sei $(N_1, \ldots, N_k)$ Zufallsvariablen, die die Kommunikationssequenz repräsentieren.
Falls ein effizienter probabilistischer Algorithmus $S$ (Simulator) existiert der 
ebenfalls solche Sequenzen produziert (gleiche Verteilung) ohne $x$ zu kennen,
dann ist das Protkoll ein Zero Knowledge-Protokoll.

\section{Graphisomorphie}
\begin{enumerate}
    \item
        $P$ erzeugt einen (großen) Zufallsgraphen \pub{$G_0$} sowie
        eine Permutation \priv{$\pi$}. Und berechnet damit \pub{$G_1 = \pi(G_0)$}.
    \item
        Mit einer weiteren Zufallspermutation \priv{$\sigma$} erzeugt $P$
        \pub{$H = \sigma{G_1}$}, dieser wird an $V$ gesendet (Commitment).
    \item
        $V$ antwortet mit einem Zufallsbit \pub{$b \in \{0, 1\}$} (Challenge).
    \item
        $P$ antwortet mit der Permutation, die $G_b$ nach $H$ überführt. Also
        \pub{$\sigma$} falls $b=1$ und \pub{$\pi \circ \sigma$} falls $b=0$.
        $V$ überprüft die Permutation.
\end{enumerate}

\section{Zero Knowledge basierend auf einer Einwegfunktion}
Wenn eine Einwegfunktion $f$ und zwei Rechenoperationen $\circ$ und $\diamond$ existieren,
so dass
\begin{equation}
    f(x \circ y) = f(x) \diamond f(y)
\end{equation}
dann kann daraus ein Zero-Knowledge Protokoll konstruiert werden:

\begin{enumerate}
    \item Initialisierung:
        $P$ wählt \priv{$x$} und berechnet \pub{$\tilde{x} = f(x)$}.
    \item Commitment:
        $P$ wählt \priv{$y$} und sendet \pub{$\tilde{y} = f(y)$} an $V$.
    \item Challenge:
        $V$ wählt ein Zufallsbit \pub{$b$} und sendet dieses an $P$.
    \item Response:
        
        Falls $b=0$: P sendet \pub{$y$}, es wird überprüft $f(y) = \tilde{y}$.

        Falls $b=1$: P sendet \pub{$z=x \circ y$}, es wird überprüft 
        $f(z) = \tilde{x} \diamond \tilde{y}$.
\end{enumerate}

\subsection{Exponentiation/Diskreter Logarithmus}
\begin{equation}
    f(x) = a^x \mod n
\end{equation}
$n$ Primzahl, $a$ Primitivwurzel. $\circ$ entspricht der Addition $\mod n-1$, und
$\diamond$ die Multiplikation $\mod n$.

\subsection{Quadrieren/Quadratwurzelziehen (Fiat-Shamir)}
\begin{equation}
    f(x) = x^2 \mod n
\end{equation}
mit $n=p \cdot q$ geheimen Primzahlen. $\circ$ und $\diamond$ sind Multiplikation $\mod n$.
