\chapter{Shannon-Entropie und absolute Sicherheit}
\section{Shannon-Entropie}
Die Shannon Entropie einer diskreten Zufallsverteilung $p$ ist definiert als:
\begin{equation}
    H(p) = -\sum_i p_i \log_2 (p_i)
\end{equation}

\section{Eigenschaften der Entropie}
\begin{itemize}
    \item 
        Die gemeinsame Entropie zweier Zufallsvariablen ist:
        \begin{equation}
            H(X, Y) = - \sum_{x, y} P(X=x, Y=y) \log_2 \left( P(X=x, Y=y)\right)
        \end{equation}
        Es gilt $H(X, Y) \leq H(X) + H(Y)$, Gleichheit gilt wenn $X$ und $Y$ unabhängig sind.
    \item
        Die bedingte Entropie ist:
        \begin{equation}
            H(X|Y) = - \sum_x P(X=x|E) \log_2\left(P(X=x | E)\right)
        \end{equation}
        bzw.
        \begin{equation}
            H(X|Y) = \sum_y P(Y=y) H(X|Y=y)
        \end{equation}
        Falls die Variablen vollständig abhängig sind gilt $H(X|Y)=0$, bei Unabhängigkeit gilt $H(X|Y) = H(X)$. Allgemein gilt $H(X|Y) \leq H(X)$.

    \item Beziehungen zwischen den Entropien:
        \begin{eqnarray}
            H(X, Y) &=& H(X) + H(Y|X) \\
            H(X|Y) &=& H(X, y) - H(Y)
        \end{eqnarray}
\end{itemize}

\section{Entropie in Kryptosystemen}
Die Nachricht $M$, der Schlüssel $K$ und die Chiffre $C$ sind Zufallsvariablen, 
dann soll gelten:
\begin{itemize}
    \item $M$ und $K$ sind unabhängig
    \item $K$ ist gleichverteilt auf einer Grundmenge
    \item $C$ wird eindeutig durch $M$ und $K$ bestimmt
    \item $M$ wird eindeutig durch $C$ und $K$ bestimmt
\end{itemize}

\section{Absolute Sicherheit}
Ein Kryptosystem ist absolut sicher, falls gilt:
\begin{equation}
    H(M) = H(M|C)
\end{equation}
D.h. die Kenntniss von $C$ bringt keine Information über $M$.

\section{One-Time-Pad}
Für einen Schlüssel $k$ der selben Länge wie die Nachricht kann das Verschlüsseln als 
bijektive Abbildung formuliert werden. Das heißt der Schlüssel kann auch aus
$m$ und $c$ berechnet werden.

Das One-Time-Pad ist absolut Sicher.

\section{Visuelle Kryptographie}
Implementierung eines One-Time-Pad mithilfe von Folien die mit einem Muster bedruckt sind, so dass sie Teilweise transparent sind.
