\chapter{Algorithmen der Kryptographie}
\section{Grundrechenarten}
$k$ sei die Länge der Eingabezahlen
\begin{itemize}
    \item \textbf{Addition/Subtraktion:} $\mathcal{O}(k)$
    \item \textbf{Multiplikation:} $\mathcal{O}(k^2)$
    \item \textbf{Division:} $\mathcal{O}(k^2)$
\end{itemize}

\section{Euklid-Algorithmus}
Laufzeit $\mathcal{O}(k^2)$.
\begin{python}
def ggT(a, b):
    if b == 0:
        return a
    else:
        return ggT(b, a % b)
\end{python}

\section{Erweiterter Euklid-Algorithmus}
Außer $d = \text{ggT}(a, b)$ wird noch $x, y$ bestimmt mit $ax + by = d$.

Laufzeit ebenfalls $\mathcal{O}(k^2)$.

\begin{python}
def extggT(a, b):
    if b == 0:
        return (a, 1, 0)
    (d, x, y) = extggT(b, a % b)
    return (d, y, x - (a // b) * y)
\end{python}

\section{Multiplikativ Inverses bestimmen}
Sei $a \in \mathbb{Z}_n^*$, dann liefert $\text{extggT}(n, a) = (1, x, y)$. Hierbei
ist $y$ das Multiplikativ inverse zu $a$.

\section{Modulare Exponentiation}
Es wird $a^b \mod n$ berechnet, $b = (b_k, \ldots, b1, b0)$ sei die Binärdarstellung von
$b$.

Laufzeit $\mathcal{O}(k^3)$.

\begin{python}
def modexp(a, b, n):
    d = 1
    for i in range(k, -1, -1):
        d = (d * d) % n
        if b_i == 1:
            d = (d * a) % n
    return d
\end{python}

\section{Chinesischer Restsatz-Algorithmus}
Gegeben $n_1, n_2, \ldots, n_k$ und $a_1, a_2, \ldots, a_k$ wie oben beschrieben.
Berechne $n = \prod_i n_i$, $m_i = \frac{n}{n_i}\ i \in \{1, \ldots, k\}$ und
$y_i$ mit $m_i \cdot y_i \equiv 1 \mod n_i$.

Dann gilt:
\begin{equation}
    x = \left( \sum_{i=1}^k a_i \cdot y_i \cdot m_i \right) \mod n
\end{equation}

\section{Miller-Rabin-Primzahltest}
Bei Ausgabe 0 $n$ keine Primzahl, bei Ausgabe $1$ mit $75\%$ Wahrscheinlichkeit Primzahl.
Durch wiederholtes Anwenden kann Wahrscheinlichkeite beliebig gesenkt werden.

Laufzeit $\mathcal{O}{k^3}$.

\begin{python}
def primtest(n):
    a = random.randint(1, n-1)
    b = n-1 # Mit Bits b_k, ..., b_0
    d = dd = 1
    for i in range(k, -1, -1):
        d = (d * d) % n
        if d == 1 and dd != 1:
            return 0
        if b_i == 1:
            d = (d*a) % n
        dd = d
\end{python}

\section{Zufallsprimzahl herstellen}
Es soll eine $k$-Bit Zufallsprimzahl hergestellt werden. Das erste und letzte Bit wird
auf $1$ gesetzt \todo{wenn ich das erste Bit auf 1 setzte wird effektiv doch nur eine
k-1 Bit Zufallsprimzahl hergestellt...} die restlichen $k-2$ Bits werden zufällig gewählt.
Dann wird ein Primmzahltest durchgeführt.

\section{Quadratische Reste modulo einer Primzahl}
Sei $n$ eine Primzahl, dann kann mit dem Euler-Kriterium geprüft werden, ob $x$ ein
quadratische Rest ist.

Sofern der Test zutrifft sollen die Quadratwurzeln berechnet werden. Falls
$n \equiv 3 \mod 4$ gilt geht das mit:
\begin{python}
def quadwurzel3(x, n):
    return modexp(x, (n+1)/4, n)
\end{python}
