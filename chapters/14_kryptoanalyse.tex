\chapter{Algorithmen der Kryptoanalyse}
\section{Pollard-$(p-1)$ (Faktorisierung)}
\begin{python}
def pollard_pminus1(n):
    a = 2
    B = 1
    while True:
        B += 1
        a = modexp(a, B, n)
        g = ggT(a-1, n)
        if g > 1:
            return g
\end{python}

\section{Pollard-$\rho$ (Faktorisierung)}
Es gilt $f(x) = x^2 + 1 \mod n$.
\begin{python}
def pollard_rho(n):
    x = random.randint(2, n-1)
    y = x
    while True:
        x = f(x)
        y = f(f(y))
        g = ggT(x-y, n)
        if g > 1:
            return g
\end{python}

\section{Fermat-Faktorisierung}
\begin{python}
def fermat_faktor(n):
    x = math.ceil(math.sqrt(n))
    z = x * x - n
    while x != n:
        if z Quadratzahl mit y^2 = x:
            return (x-y, x+y)
        x *= y
        z += 2*x +1
\end{python}

\section{Giant-Step-Baby-Step-Algorithmus (D-Log)}
Ausgangspunkt:
\begin{equation}
    y \cdot {(a^{-1})}^{x_2} = {\left(a^{\ceil{\sqrt{n}}}\right)}^{x_1}
\end{equation}
Mit $u=a^{-1}$ und $v=a^{\ceil{\sqrt{n}}}$ füllen wir die folgende Liste:
\begin{equation}
    L = \{(x_1, v_1^{x_1}) | x_1 = 0, \ldots \floor{\sqrt{n}} \}
\end{equation}
nun können wir ein $x_1, x_2$ paar und daraus $x$ bestimmen:
\begin{python}
    for x_2 in range(0, math.ceil(math.sqrt(n))):
        if y * u**x_2 in L:
            x_1, _ = L[y * u**x_2]
            return (x_1 * math.ceil(math.sqrt(n)) + x_2)
\end{python}

\section{Pollary-$\rho$ (D-Log)}
Wir erzeugen eine Pseudozufallssequenz $z_k = a^{i_k} y_{j_k}$, nach $\sqrt{n}$ Schritten
finden wir (im Erwartungswert) ein paar $z_\mu = z_\sigma$. Da $y=a^x$ gilt muss dann gelten:
\begin{equation}
    i_\mu + j_\mu \cdot x = i_\sigma + j_\mu \cdot x
\end{equation}
woraus sich $x$ berechnen lässt. Für den Zufallsgenerator zerlege $M=\{0, \ldots, n-1\}$
in drei (etwa) gleichgroße disjunkte Teilmengen $M_1, M_2, M_3$.
Gegeben $z=a_i \cdot y_i$, dann wird je nach Menge in der $z$ ist unterschieden:
\begin{enumerate}
    \item
        $i' = i + 1 \mod n-1$
    \item
        $j' = j + 1 \mod n-1$
    \item
        $i' = i \cdot 2 \mod n-1$ und $j' = j \cdot 2 \mod n-1$
\end{enumerate}
