\chapter{Klassische Kryptographie}
\section{Pseudozufallsgeneratoren, rückgekoppelte Schieberegister}
Der Ansatz eines One-Time-Pad ist in den meisten Anwendungen praktisch nicht realisierbar.
Daher wird ein Verfahren benötigt, dass einen Schlüssel vergrößert, dieses Verfahren
ist nicht absolut sicher, kann aber ein beliebiges Sicherheitsniveau erreichen.

Mögliche Verfahren:
\begin{itemize}
    \item Linearer Kongruenzgenerator:
        \begin{equation}
            z_{i+1} = (a \cdot z_i + b) \mod n
        \end{equation}
        für eine maximale Periode muss gelten: $b, n$ teilerfremd, $a \equiv 1 \mod p$ für alle Primteiler $p$ von $n$. Falls $n$ durch $4$ teilbar ist muss auch $a \equiv 1 \mod 4$ gelten.
    \item Polynome höheren Grades, z.B.:
        \begin{equation}
            z_{i+1} = a \cdot z_i^2 + b \cdot z_i + c \mod n
        \end{equation} 
    \item Blum-Blum-Shub-Generator: wähle Primzahlen $p, q$, $n=p\cdot q$        
        \begin{equation}
            z_{i+1} = z_i^2 \mod n
        \end{equation}
    \item Rückgekoppelte Schieberegister: 
        Die Ausgabe des Schieberegisters wird mit anderen Zuständen (XOr-) Verknüpft,
        bei jedem Taktzyklus ändert sich der Zustand. Für $n$ Zustände erreicht man eine
        maximale Periode von $2^n-1$ ($0$ ist kein möglicher Zustand). Verbesserte
        Sicherheit bei Verknüpfung von mehreren Schieberegistern unterschiedlicher
        Länge.
\end{itemize}

\section{Feistel-Netzwerk, DES}
Netzwerk mit mehreren Runden, DES unsicher, da Schlüssel zu kurz.

Aus dem Schlüssel $K$ werden $n$ Rundenschlüssel generiert. Die Nachricht wird in zwei
gleich lange Blöcke aufgeteilt $m=(L_0, R_0)$. Dann werden $n$ Rundendurchlaufen, wobei
jeweil $(L_{i+1}, R_{i+1})$ aus $(L_i, R_i)$ und $K_i$ berechnet werden. Die verschlüsselte
Nachricht ist $(L_n, R_n)$.

Die Berechnung erfolgt durch (die Bestimmung von $f$ und $K$ wird nicht behandelt):
\begin{eqnarray}
    L_i &=& R_{i-1} \\
    R_i &=& L_{i-1} \oplus f_i(R_{i-1}, K_i)
\end{eqnarray}
die Dechiffrierung erfolgt durch das selbe Schema in umgedrehter Richtung.

\section{Modulares Potenzieren}
Es wird eine Primzahl $n$ benötigt sowie ein $e$ mit $\text{ggT}(n-1, e) = 1$. Damit kann
das $d$ mit $e \cdot d \equiv 1 \mod n-1$ berechnet werden.
Für die Verschlüsselung gilt:
\begin{eqnarray}
    c &=& m^e \mod n \\
    m &=& c^d \mod n \\
\end{eqnarray}
