\chapter{Historische Chiffren}
\section{Cäsar-Chiffre}
Einfache Substitutionschiffre, bei der jeder (numerische) Buchstabenwert um ein gewisses Offset verschoben wird.

\section{Affine Chiffre}
Der Schlüssel besteht aus zwei Zahlen $k=(a,b)$, verschlüsselt wird durch:
\begin{equation}
    c= a \cdot m + b \mod 26
\end{equation}
Für die injektivität muss aber gelten $\text{ggT}(a, 26) = 1$.

\section{Allgemeine monoalphabetische Chiffrierung}
Jeder der 26 Buchstaben kann auf einen anderen Buchstaben abgebildet werden. Es existieren somit $26! \approx 2^{88}$ Schlüssel. Das Sicherheitsniveau kann trotzdem nicht gegeben sein, wenn z.B. Buchstabenhäufigkeiten ausgenutzt werden können.

\section{Buchstabenhäufigkeiten im Deutschen}
\subsection{Einzelbuchstaben}
Grobe Reihenfolge: E (17\%), N (10\%), I, R, S, T, A (jeweils circa $6-7\%$).

\subsection{Buchstaben-Paare (Bigramme)}
Häufig sind: ER, EN, CH, DE, EI, ND, TE, IN, IE.
G
\section{Homophone Chiffrierung}
Ein Quellensymbol wird auf mehrere anderes Symbole zufällig abgebildet, die Anzahl der
jeweiligen Codesymbole ist proportional zur Wahrscheinlichkeit.
Dadurch wird eine Gleichverteilung von Codewörtern erreicht.
Immer noch angreifbar über Bi- und Trigramme.

\section{Playfair-Chiffre}
Es werden Bigramme auf Bigramme abgebildet. Doppelte Buchstaben sollten vermieden werden,
z.B. durch Streichen der Dopplung oder Einfügen eines Füllbuchstabens. Außerdem wird der
Buchstabe J durch I ersetzt.

Das Schlüsselwort hat meistens die Länge 5 und sollte keine doppelt vorkommendende 
Buchstaben haben. Es wird eine $5 \times 5$ Tabelle reihenweise aufgefüllt, hierfür
wird mit dem Schlüsselwort angefangen und dann nachfolgend das (modifizierte) Alphabet
eingetragen wobei Buchstaben die bereits Teil des Schlüsselwortes sind ausgelassen werden.

Das Verschlüsseln erfolgt auf Basis der Tabelle:
\begin{itemize}
    \item Wenn beide Buchstaben in der selben Zeile oder Spalte liegen, dann werden 
        jeweils die nachfolgenden Buchstaben (in Zeile bzw. Spalte) zyklisch ausgewählt.
    \item Sonst bilden die Buchstaben die Ecken eines Rechtecks, es werden im Uhrzeigersinn
          die anderen Buchstab des Rechtecks ausgewählt.
\end{itemize}

Auch dieses Verfahren ist über Bi- und Trigramme angreifbar.

\section{Vigenère-Chiffre}
Man wählt ein Schlüsselwort, welches fortlaufend über den Klartext geschrieben wird, dann wird Buchstabenweise-Caesar verschlüsselt.

\subsection{Kasiski}
Mehrfachvorkommen von Trigrammen finden und deren Differenzen finden. 
Die wahrscheinliche Schlüssellänge ist dann der ggT aller Differenzen.
Dann können die einzelnen Caesar-Chiffren über die Buchstabenhäufigkeiten geknackt werden.

\section{Autokey-Chiffre}
\subsection{Variante 1}
Modifikation der Vigenère-Chiffre bei der statt dem wiederholten Schlüsselwort der letzte verschlüsselte Block verwendet wird.

\subsection{Variante 2}
Modifikation der Vigenère-Chiffre bei der statt dem wiederholten Schlüsselwort der letzte Klartext-Block verwendet wird.

Beide Verfahren sind ebenfalls unsicher.

