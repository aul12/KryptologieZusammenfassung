\chapter{Elliptische Kurven}
Eine elliptische Kurve wird über einen Körper $\mathbb{K}$ und den Parametern
$a, b$ definiert:
\begin{equation}
    E_{a, b} = \{(x, y) \in \mathbb{K}^2 | y^2 = x^3 + a \cdot x + b\}
\end{equation}
Diese Menge (mit $\mathcal{O}$, siehe unten) bildet bezüglich der Operation 
$\diamond$ eine Gruppe. $\mathcal{O}$ ist das neutrale Element.

Hierbei ist $P \diamond Q$ definiert als:
\begin{itemize}
    \item Es wird eine Gerade durch $P$ und $Q$ gelegt, diese schneidet die Kurve in
        einem dritten Punkt $R$, durch spiegeln an der $x$ Achse erhalten wir $R'$
        es gilt: $P \diamond Q = R'$.
    \item Falls $P=Q$ wird nicht die Verbindungsgerade sondern die Tangente in $P$
        genommen.
    \item Falls $P$ und $Q$ senkrecht übereinander liegen so wird ein neuer Punkt
        $\mathcal{O}$ eingeführt, dieser liegt im unendlichen.
\end{itemize}


Statt des diskreten Logarithmus können auch Operation auf elliptischen Kurven über
$\mathbb{Z}_p$ mit $p$ Prim genutzt werden. Hierbei wird ein ähnliches Sicherheitsniveau
bei kleineren Schlüsseln erreicht.
