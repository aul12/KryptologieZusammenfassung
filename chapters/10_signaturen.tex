\chapter{Signaturen}
Einer Nachricht $m$ wird eine Unterschrift $U$ von $A$ hinzugefügt, dabei muss gelten:
\begin{itemize}
    \item
        Es soll überprüfbar sein, dass $u$ von $A$ erstellt wurde
    \item
        Nur $A$ soll solch eine Unterschrift herstellen können
    \item
        Das Dokument und die Unterschrift sollen nicht so abgeändert werden könnnen,
        dass es ausssieht als ob ein anderes Dokument unterschrieben wurde
\end{itemize}

\section{Public Key-Infrastruktur}
Wenn der Teilnehmer $A$ das Schlüsselpaar \pub{$k_A$}, \priv{$k_A'$} besitzt, so kann
gelten:
\begin{equation}
    u = D(k_A', m)
\end{equation}
Hinweis: hier muss gelten $E(k_A, D(k_A', m)) = m$, also Verschlüsselung und 
Entschlüsselung müssen vertauscht werden können, bei RSA ist das trivialerweise der
Fall.

\section{Kurze Unterschriften mit Hashing}
Um Signaturen zu verkleinern kann statt $m$ auch $h(m)$, mit $h$ einer Hashfunktion,
signiert werden. Die Hashfunktion sollte Kollisionsresistent sein.

\section{Geburtstagsproblem}
Angriffe die Kollisionen erreichen wollen sind effizienter als erwartet, bei $n$ möglichen
Hashwerten müssen nur circa $\sqrt{n}$ Nachrichten getestet werden bis im Erwartungswert
die erste Kollision auftritt.

Also muss der Hash entsprechend mindestens doppelt so lang wie das geforderte 
Sicherheitsniveau sein.

\section{Blinde Unterschrift mit RSA}
Problem: Bob soll Dokument von Alice unterschreiben, ohne das Dokument zu sehen.

\begin{itemize}
    \item
        Alice wählt Zufallszahl \priv{$z \in \mathbb{Z}_n^*$} und verblendet
        \priv{$m$} zu \pub{$m' = z^e \cdot m \mod n$}. \pub{$(e, n)$} ist Bobs
        öffentlicher Schlüssel.
    \item
        Bob unterschreibt, berechnet also \pub{${(m')}^d \mod n$} mit seinem
        privaten Schlüssel \priv{$d$}.
    \item Alice multipliziert mit \priv{$z^{-1}$} und erhält $u=m^d \mod n$
\end{itemize}
