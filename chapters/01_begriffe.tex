\chapter{Begriffe, Basisdefinitionen}
\section{Kryptologie}
Kryptologie umfasst:
\begin{itemize}
    \item Kryptographie (Ver- und Entschlüsselung)
    \item Kryptoanalyse (Analyse der Sicherheit)
\end{itemize}

\section{Klassische/Symmetrische Kryptographie}
\begin{itemize}
    \item $m$: Nachricht (\textit{Message})
    \item $k$: Schlüssel (\textit{Key})
    \item $c=E(m,k)$: Chiffre ($E \hat{=}$ \textit{Encrypt}) 
    \item $m=D(c,k)$: Entschlüsselte Nachricht ($D \hat{=}$ \textit{Decrypt})
\end{itemize}
$E$ muss injektiv bezüglich des ersten Arguments sein um Entschlüsselung zu ermöglichen (unterschiedliche Chiffren).
Der Schlüsseltausch ist nicht spezifiziert und muss extern geschehen.

\section{Transpositions- und Substitutionschiffren}
Bei Transpositionschiffren bleiben die Buchstaveb unverändert, nur ihre Reihenfolge wird verändert.

\section{Monoalphabetischechiffren}
Bei Monoalphabetischen Chiffren wird jedes Quellen Symbol fest auf ein anderes Symbol abgebildet.

\section{Polyalphabetischechiffren}
Es wird Buchstabenweise ersetzt, die Abbildung ändert sich jedoch nach einem festgelegten Schema.

\section{Blockchiffren}
Blöcke fester Länge werden in Chiffrenblöcke abgebildet.

\section{Lawineneffekt}
Wünschenswert ist das bereits kleine Änderungen am Klartext große Änderungen an der Chiffre hervorrufen.
Monoalphabetischesubstitutionschiffren und Substitutionschiffren erfüllen diese Anforderung nicht.

\section{Moderne/Asymmetrische Kryptographie}
Der Empfänger erzeugt zwei Schlüssel, den öffentlichen Schlüssel $k$ und den privaten Schlüssel $k'$. 
Es gilt: $E(m, k) = c$ und $D(c, k') = m$, außerdem soll es für einen
Angreifer nicht möglich sein $k'$ aus $k$ zu berechnen.
Hier ist also kein Schlüsseltausch notwendig.

