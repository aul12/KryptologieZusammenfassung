\chapter{Begriffe, Basisdefinitionen}
\section{Kryptologie}
Kryptologie umfasst:
\begin{itemize}
    \item Kryptographie (Ver- und Entschlüsselung)
    \item Kryptoanalyse (Analyse der Sicherheit)
\end{itemize}

\section{Klassische/Symmetrische Kryptographie}
\begin{itemize}
    \item $m$: Nachricht (\textit{Message})
    \item $k$: Schlüssel (\textit{Key})
    \item $c=E(m,k)$: Chiffre ($E \hat{=}$ \textit{Encrypt}) 
    \item $m=D(c,k)$: Entschlüsselte Nachricht ($D \hat{=}$ \textit{Decrypt})
\end{itemize}
$E$ muss injektiv bezüglich des ersten Arguments sein um Entschlüsselung zu ermöglichen (unterschiedliche Chiffren).
Der Schlüsseltausch ist nicht spezifiziert und muss extern geschehen.

\section{Transpositions- und Substitutionschiffren}
Bei Transpositionschiffren bleiben die Buchstaveb unverändert, nur ihre Reihenfolge wird verändert.

\section{Monoalphabetischechiffren}
Bei Monoalphabetischen Chiffren wird jedes Quellen Symbol fest auf ein anderes Symbol abgebildet.

\section{Polyalphabetischechiffren}
Es wird Buchstabenweise ersetzt, die Abbildung ändert sich jedoch nach einem festgelegten Schema.

\section{Blockchiffren}
Blöcke fester Länge werden in Chiffrenblöcke abgebildet.

\section{Lawineneffekt}
Wünschenswert ist das bereits kleine Änderungen am Klartext große Änderungen an der Chiffre hervorrufen.
Monoalphabetischesubstitutionschiffren und Substitutionschiffren erfüllen diese Anforderung nicht.

\section{Moderne/Asymmetrische Kryptographie}
Der Empfänger erzeugt zwei Schlüssel, den öffentlichen Schlüssel $k$ und den privaten Schlüssel $k'$. 
Es gilt: $E(m, k) = c$ und $D(c, k') = m$, außerdem soll es für einen
Angreifer nicht möglich sein $k'$ aus $k$ zu berechnen.
Hier ist also kein Schlüsseltausch notwendig.

\section{Algorithmen, Protokolle und Runden}
Ein Algorithmus beschreibt die Berechnung eines Wertes aus anderen Werten, hierbei kann ein Algorithmus deterministisch oder auch probabilistisch sein.
Ein Protokoll spezifiert den Nachrichtenaustausch, wenn mehrfach (verschiedene) Nachrichten versendet werden, so spricht man von Runden.

\section{Hybride Kryptosysteme}
Da asymmetrische Kryptographie einen höheren Aufwand (circa $\mathcal{O}(n^3)$) im 
Vergleich zu symmetrischen Verfahren (circa $\mathcal{O}(n)$) hat werden symmetrische
und asymmetrische Verfahren kombiniert. Für den Schlüsseltausch werden asymmetrische
Verfahren verwendet, die eigentliche Kommunikation nutzt symmetrische Verfahren.

\section{Angriffsszenarien}
\begin{itemize}
    \item
        \textbf{Cyphertext-only:} Angreifer kennt nur $c$ und soll $m$ (und eventuell auch $k$) ermitteln
    \item 
        \textbf{Known-Plaintext:} Angreifer kennt $(m, c)$ (evtl. mehrere) und soll $k$ ermitteln
    \item
        \textbf{Chosen-Plaintext:} Angreifer kann $c=E(m, k)$ für beliebiges $m$ berechnen und soll daraus $k$ bestimmen
    \item
        \textbf{Chosen-Cyphertext:} Angreifer kann $m=D(c,k)$ für beliebiges $c$ berechnen und soll daraus $k$ bestimmen
    \item
        \textbf{Man-in-the-middle:} Angreifer kann Nachrichten zwischen Alice und Bob verändern oder austauschen.
        Ohne zusätzliche Authentisierung ist ein solcher Angriff nicht vermeidbar.
\end{itemize}

\section{Kerckhoff-Prinzip}
Die Sicherheit eines Verfahrens sollte auf der Geheimhaltung der Schlüssel und nicht des
Verfahrens basieren.

\section{Sicherheitsniveau}
Das Sicherheitsniveau vergleicht ein Verfahren mit einem äquivalenten, nur durch Brute-Force lösbaren Verfahren mit Schlüssellänge $n$. Das heißt bei zweiteren Verfahren müssen
genau $2^n$ Schlüssel probiert werden, bei einem Verfahren mit Sicherheitsniveau $n$ ist
der Aufwand genauso groß.
