\chapter{P, NP, Einwegfunktionen und Orakel}
\section{Die Klasse P}
Umfasst alle algorithmischen Aufgabenstellungen für die es einen (deterministischen)
Algorithmus gobt, dessen Rechnezeit als Funktion der Eingabelänge polynomial ist.

Polynomial wird vereinfacht auch als effizient berechenbar bezeichnet.

\section{NP und NP-vollständig}
Die Klasse NP ist für Entscheidungsprobleme definiert: Wir können das
Problem selber im Allgemeinen nicht in polynomialer Zeit lösen, können
aber in polynomialer Zeit bestimmen ob eine gegebene Lösung korrekt ist.

NP vollständige Probleme sind solche für die (wahrscheinlich) keine
polynomiale Lösung existiert.

\section{Einwegfunktionen}
Eine Einwegfunktion ist eine Fumktion, welche nur in Vorwärtsrichtung effizient
berechenbar ist:
\begin{equation}
    f \in \text{P aber } f^{-1} \notin \text{P}
\end{equation}
Bei nicht-injektiven Funktionen (z.B. Hashfunktionen) soll kein $x = f^{-1}(y)$ effizient
berechnet werden können.

In der Kryptographie soll außerdem die Umkehrfunktion auch nicht mit probabilitistischen
Algorithmen effizient zu berechnen sein.

\section{Orakel-Reduzierbarkeit}
Ein Problem $A$ ist auf ein anderes Problem $B$ effizient Orakel reduzierbar 
($A \preceq B$) falls es einen polynomialen Algorithmus gibt, der $A$ löst.
Dieser Algorithmus  darf das Lösen von $B$ als Unterprogram nutzen, wobei
die Komplexität von $B$ nicht in die Gesamtkomplexität einfließt.

\todo{Selbst-Reduzierbarkeit von DLog
konkrete Reduktionen: alle solche, die Äquivalenz zu Faktorisieren zeigen}
